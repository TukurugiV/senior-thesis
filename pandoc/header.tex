% 論文向けLaTeXヘッダー
% カスタム環境の定義

% セクション番号を「第n章」形式に設定(追加パッケージ不要)
\renewcommand{\thesection}{第\arabic{section}章}
\renewcommand{\thesubsection}{\arabic{section}.\arabic{subsection}}
\renewcommand{\thesubsubsection}{\arabic{section}.\arabic{subsection}.\arabic{subsubsection}}

% 目次の「第n章」がかぶらないように幅を調整(標準コマンド使用)
\makeatletter
\renewcommand{\l@section}{\@dottedtocline{1}{0em}{4.5em}}
\renewcommand{\l@subsection}{\@dottedtocline{2}{1.5em}{3.5em}}
\renewcommand{\l@subsubsection}{\@dottedtocline{3}{3.8em}{4.5em}}
\makeatother

% 段落の1行目を字下げ(セクション直後も含む)
\usepackage{indentfirst}
\usepackage{etoolbox}
\makeatletter
\let\@afterindentfalse\@afterindenttrue
\@afterindenttrue
\makeatother

% 必要なパッケージ
\usepackage{amsthm}
\usepackage{graphicx}    % includegraphics コマンド用
\usepackage{caption}     % captionof コマンド用
\usepackage{subcaption}  % 画像の横並び・サブキャプション用
\setcounter{secnumdepth}{3} % 章番号を表示

% 数式番号をセクションごとにリセット
\numberwithin{equation}{section}

% 図番号をセクションごとにリセット
\numberwithin{figure}{section}

% 表番号をセクションごとにリセット
\numberwithin{table}{section}

% 日本語用の環境名定義
\theoremstyle{definition}

% 定理環境
\newtheorem{theorem}{定理}[section]

% 補題環境
\newtheorem{lemma}{補題}[section]

% 定義環境
\newtheorem{definition}{定義}[section]

% 例環境
\newtheorem{example}{例}[section]

% アルゴリズム環境
\newtheorem{algorithm}{アルゴリズム}[section]

% 注釈環境(番号なし)
\theoremstyle{remark}
\newtheorem*{note}{注}

% 警告環境(番号なし)
\newtheorem*{warning}{警告}

% 証明環境(amsthmのデフォルトを使用、日本語化)
\renewcommand{\proofname}{証明}
