% 論文向けLaTeXヘッダー
% カスタム環境の定義

% コードブロック内の日本語表示用フォント設定
\setmonofont{BIZ UDGothic}

% セクション番号を「第n章」形式に設定(追加パッケージ不要)
\renewcommand{\thesection}{第\arabic{section}章}
\renewcommand{\thesubsection}{\arabic{section}.\arabic{subsection}}
\renewcommand{\thesubsubsection}{\arabic{section}.\arabic{subsection}.\arabic{subsubsection}}

% 目次の「第n章」がかぶらないように幅を調整(標準コマンド使用)
\makeatletter
\renewcommand{\l@section}{\@dottedtocline{1}{0em}{4.5em}}
\renewcommand{\l@subsection}{\@dottedtocline{2}{1.5em}{3.5em}}
\renewcommand{\l@subsubsection}{\@dottedtocline{3}{3.8em}{4.5em}}
\makeatother

% 段落の1行目を字下げ(セクション直後も含む)
\usepackage{indentfirst}
\usepackage{etoolbox}
\makeatletter
\let\@afterindentfalse\@afterindenttrue
\@afterindenttrue
\makeatother

% 必要なパッケージ
\usepackage{graphicx}    % includegraphics コマンド用
\usepackage{caption}     % captionof コマンド用
\usepackage{subcaption}  % 画像の横並び・サブキャプション用
\setcounter{secnumdepth}{3} % 章番号を表示

% 表のキャプションを上に配置(日本語論文の慣例)
\captionsetup[table]{position=above, skip=5pt}
